\textbf{Background.}
Two required parameters for mathematical models of sexually transmitted infections are
the mean duration in epidemiological risk states (\eg selling sex) and
the mean rates of sexual partnership change.
While much attention has been paid to
sampling and reporting biases affecting these parameters,
relatively little work has examined measurement error
in quantifying dynamic sexual behaviour from cross-sectional data.
\textbf{Methods.}
We explore adjustments for several biases affecting aggregate estimates of
duration in sex work and numbers of reported sexual partners
from a published 2011 survey of female sex worker in Eswatini.
We develop adjustments from first principles,
and construct Bayesian hierarchical models to reflect
our mechanistic assumptions about the bias-generating processes.
\textbf{Results.}
We show that different mechanisms of bias for duration in sex work may
``cancel out'' by acting in opposite directions,
but that failure to consider some mechanisms could over/underestimate
duration in sex work by factors approaching 2.
We also show that conventional interpretations of sexual partner numbers
are biased due to implicit assumptions about partnership duration,
but that unbiased estimators of partnership change rate can be defined
that explicitly incorporate a given partnership duration.
We highlight how the unbiased estimator is most useful when
the survey recall period and partnership duration are similar in length.
\textbf{Conclusions.}
While we explore these bias adjustments using one particular dataset,
and in the context of deriving inputs for mathematical modelling,
we expect that our approach and insights would be applicable to
other datasets and motivations for quantifying sexual behaviour data.

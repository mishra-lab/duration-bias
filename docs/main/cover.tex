\address{
  Dr. Samuel Jenness\\
  Editor\\
  Epidemiology
}{Dr. Sharmistha Mishra\\
  MAP Centre for Urban Health Solutions\\
  St. Michael's Hospital, Unity Health Toronto\\
  University of Toronto}
Dear Dr. Jenness,
\par
We are pleased to submit the attached manuscript entitled
\emph{Adjusting for duration biases in sexual behaviour data}
for consideration as a \emph{Validation} study in \emph{Epidemiology}.
\par
Quantitative estimates of sexual behaviour are required
as inputs to mathematical models of sexually transmitted infections,
and in other studies of sexually transmitted infection epidemiology.
Such estimates are often derived from cross-sectional surveys.
While previous work has explored biases associated with survey-based estimates
(\eg rounding error, recall bias, reporting bias),
less attention has been paid to biases associated with estimate-estimand mismatch
% SM: do we use estimate-estimand in the intro? i may have missed it sorry!
%     but i agree with this approach re: biases associated with estimate-estimand mismatch :)
% TODO
(\eg numbers of partners in the past 30 days \vs partnership change rate).
\par
In this study, we explore precise interpretation of survey data to inform two key variables:
durations in epidemiological risk states (\eg selling sex) and
rates of sexual partnership change (\eg casual partners per year).
We identify potential sources of bias,
and develop Bayesian hierarchical models to reflect
mechanistic assumptions about the bias-generating processes.
Fitting these models to aggregate data from
a previously published study of female sex workers in Eswatini,
we show that failure to account for particular biases can
substantially influence estimates of both variables explored.
\par
While we focus on sexual behavior data,
we expect that our approach and findings would be relevant to
other estimates of intermittent risk exposure and event rates from cross-sectional surveys,
% SM: add this to the discussion/conclusion section of paper - is more concrete/precise re:
%     the "so what re: broader applicability" examples than what we currently have :)
% TODO
and thus of interest to the broad readership of \emph{Epidemiology}.
\par
Thank you for your consideration and we look forward to hearing from you.
\medskip\par
Sincerly,
\par
Jesse Knight, Siyi Wang, and Sharmistha Mishra
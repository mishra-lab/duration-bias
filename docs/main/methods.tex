\section{Methods}\label{meth}
Full details of the FSW survey methodology are available in \cite{Yam2013}.
Briefly, 328 women aged 15+
who reported exchanging or selling sex for money, favors, or goods in the past 12 months
were recruited via respondent-driven sampling (RDS) \cite{Heckathorn1997}.
\clearpage % TEMP
%---------------------------------------------------------------------------------------------------
\subsection{Risk Group Duration}\label{meth.yss}
The survey included questions about
the current respondent's age and the age of first selling sex.
The difference between these ages could be used to define a crude ``duration selling sex''.
Using this approach, the median unadjusted duration among FSW in Eswatini was 4 years.
% TODO: not reported in Baral2014
However, this estimate can be improved by considering the following potential biases.
\paragraph{Distribution}
In compartmental transmission models,
durations are implicitly assumed to be exponentially distributed.
This assumption was found to be reasonable here (see Figure~\ref{fig:yss.fit}),
but the median of an exponential distribution is less than the mean
by a factor of $\distr{log}{2}$ due to skewness.
Thus, the unadjusted mean duration could be estimated as $\bar{D}^*_s = 4/\distr{log}{2} = 5.77$.
\paragraph{Sampling}
Sampling error was considered via RDS-adjustment in \cite{Baral2014},
yielding estimates of the proportions of FSW
who had sold sex starting 0--2, 3--5, 6--11, and 11+ years ago.
The adjusted proportions indicate fewer years selling sex \vs the unadjusted proportions,
which would be consistent with
challenges in reaching women in the first year(s) of sex work \cite{Cheuk2020}.
Fitting an exponential distribution to the cumulative adjusted proportions
(Figure~\ref{fig:yss.fit})
yielded an estimated distribution mean $\bar{D}_s$ of 4.1 (95\%~CI: 3.4,~4.9) years.
% TODO: describe fitting methods (app)
\paragraph{Censoring}
The reported durations are right censored
because almost all respondents will continue selling sex after the survey \cite{Fazito2012}.
If we assume that the survey reaches FSW
at a random time point during their total (eventual) duration selling sex $D$,
then the duration reported in the survey is effectively $D_s \sim \distr{Unif}{0,D}$.
Thus, the mean duration reported in the survey is $\bar{D}_s = \frac12 \bar{D}$,
and we can define $f = \bar{D} / \bar{D}_s = 2$,
to give a further adjusted estimate as: $\bar{D} = f\bar{D}_s$.
In case the RDS-adjustment did not fully account for delayed self-identification as FSW,
we could use $f \sim \distr{Unif}{1.5,2}$, or similar.
\paragraph{Measurement}
Finally, FSW may not sell sex continuously.
The 2011 survey did not ask whether respondents ever stopped selling sex,
but a 2014 survey did, finding that $\phi = 45\%$ had stopped at least once \cite{EswKP2014}.
% TODO: not reported in EswKP2014
Among the respondents who stopped, we have no further information about
the proportion of the total period (\ie since first started selling sex)
reflected in the current period (\ie since re-starting most recently).
We denote this proportion $\rho$, and define the expected value for two extreme cases:
respondents were almost \emph{never} selling sex during the total period ($\rho = 0$), or
respondents were almost \emph{always} selling sex ($\rho = 1/3$).
Thus, we can define the final adjusted estimate for duration in sex work as:
$\bar{D} = f[(1-\phi)+(\phi)\rho]\bar{D}_s$, with
$\phi = 0.45$, $\rho \sim \distr{Unif}{0,1/3}$, and $f \sim \distr{Unif}{1.5,2}$.
%---------------------------------------------------------------------------------------------------
\subsection{Sexual Partnership Duration}\label{meth.partners}
%The FSW survey asked respondents to report
their numbers of unique sexual partners in the past 30 days,
stratified by three types of partner:
new paying clients, regular paying clients, and non-paying partners.
For illustrative purposes, we assume that
only a small proportion of new clients would go on to become regular clients;
thus, we conceptualize ``new'' clients as effectively ``one-off'' clients.
We further assume that partnership durations were:
1~day with new paying clients,
4~months with regular paying clients, and
3~years with non-paying partners
(no survey questions asked about partnership durations).
% TODO: cite?
\par
% TODO: need to derive this in code
The RDS-adjusted median reported numbers of partnerships of each type were:
1.77, 4.69, and 0.74, respectively \cite{Baral2014}.
Our aim is to use these reported numbers of partners ($x$)
for the 30-day recall period ($\omega$),
with the assumed partnership durations ($\delta$),
to define expected rates of partnership change ($Q$) and numbers of concurrent partnerships ($K$).
Here we ignore distribution assumptions entirely, since
partnership rates are assumed to be homogeneous in compartmental transmission models.
\par
We begin with some general observations about
the relationships between the variables $x, \omega, \delta, Q, K$.
If partnership duration is long and the recall period is short ($\delta \gg \omega$),
the reported partnerships mostly reflect \emph{ongoing} partnerships,
and thus $x \approx K$.
If partnership duration is short and the recall period is long ($\delta \ll \omega$),
the reported partnerships mostly reflect \emph{complete} partnerships,
and thus $x/\omega \approx Q$.
However, if partnership duration and recall period are similar in length,
the reported partnerships reflect a mixture of tail-ends, complete, and ongoing partnerships,
and thus $x$ overestimates $K$, but $x/\omega$ also overestimates $Q$.
\par
We proceed by making a similar assumption as in \sref{meth.yss}:
that survey timing is effectively random with respect to partnership duration.
Then, if either end of the recall period would capture an ongoing partnership,
the intersection point would be, on average, at the partnership mid-point.
Thus, the recall period is effectively extended
by half the partnership duration $\delta/2$ on each end, and $\delta$ overall,
as illustrated in Figure~\ref{fig:diag.partners}.
As such, we can define $Q$ and $K$ as:
\begin{alignat}{1}
  Q &= \frac{x}{\omega + \delta}\\
  K &= \frac{x \delta}{\omega + \delta} = Q \delta
\end{alignat}

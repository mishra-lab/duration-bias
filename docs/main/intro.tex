\section{Introduction}
% SM: this is good! after you revise, can I look at the intro & discussion once agaiin quickly before we submit?
Epidemic modelling of sexually transmitted infections (STI) relies on
quantification of sexual behaviour for model inputs (parameters) \cite{Garnett2002}.
In models of STI transmission with risk heterogeneity
--- \ie considering subgroups that experience differential risks ---
two important paramteters are:
the duration of time within a ``risk group'', and
the rate of sexual partnership formation, possibly stratified by partnership type
\cite{Garnett1996,Stigum1997,Watts2010,Knight2020}.
For example, the average duration of time engaged in sex work
can be used to define the modelled rate of ``turnover'' among sex workers \cite{Watts2010}.
Similarly, the numbers of main, casual, transactional, and/or paying sexual partners per year
can be used to define the modelled rate of infection incidence \cite{Boily2015}.
\par
Data to inform these parameters largely come from cross-sectional studies,
and are often only available as crude aggregate estimates (\vs individual-level data).
Such estimates may be subject to distributional, sampling, censoring, and measurement biases.
Our aim is therefore to develop bias adjuments for estimating:
\begin{enumerate}
  \item duration in a risk group
  \item rate of partnership change
\end{enumerate}
from aggregate cross-sectional survey data, considering these factors.
We explore these topics using aggregate estimates from
a 2011 female sex worker survey in Eswatini~\cite{Baral2014}.

\section{Introduction}
Mathematical models of sexually transmitted infections require
quantitative estimates of sexual behaviour for model inputs (parameters) \cite{Garnett2002}.
In such models with risk heterogeneity
--- \ie considering states that experience differential risks ---
two important parameters are:
the duration of time within an epidemiological risk state/group (or period/season of risk) and
the rate of sexual partnership change (often stratified by partnership type)
\cite{Garnett1996,Stigum1997,Watts2010,Knight2020}.
For example, the average duration of time engaged in sex work
can be used to define the modelled rate of ``turnover'' among sex workers \cite{Watts2010}.
Similarly, the numbers of main, casual, transactional, and/or paying sexual partners per year
can be used to define the modelled rate of infection incidence \cite{Boily2015}.
\par
Data to inform these parameters largely come from cross-sectional studies,
and are often only available as aggregate estimates (\vs individual-level data).
Such estimates may be subject to distributional, sampling, censoring, and measurement biases.
Our aim is therefore to explore bias adjuments for estimating:
(1) duration in a risk state, and (2) rate of partnership change,
from aggregate cross-sectional survey data.
We explore and demonstrate these topics using data from
a 2011 female sex worker survey in Eswatini~\cite{Baral2014},
to support parameterization of a mathematical model of heterosexual HIV transmission.

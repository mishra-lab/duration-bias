\section{Introduction} %SM: this is good! after you revise, can I look at the intro & discussion once agaiin quickly before we submit?
Epidemic modeling of sexually transmitted infections (STI) relies on 
quantifying inputs (parameters) surrounding sexual behaviour in the transmission model \cite{Fenton2001}.
When developing mathematical models of STI transmission in the context of risk heterogeneity (i.e.
subgroups experience differential risks),  
two important paramteters are:  %SM: nice and succint!
the duration of time within a ``risk group'' , and
the rate of sexual partnership formation, possibly stratified by partnership type
\cite{Garnett1996,Stigum1997,Henry2015,Knight2020}.
For example, in the context of sex work, parameters on duration of time engaged in sex work 
are used in the transmission model to... %SM; but would add/say something like this here for  both things, to make it easier for a reader who is not familiar with STI/HIV modeling per se...

Data to inform these important parameters largely come from cross-sectional studies. 
Furthermore, the data are often 
only available as crude aggregate estimates
from previously published studies (\vs individual-level data).
Whether available as individual-level or aggregate, crude estimates may be subject to  %SM: b/c your solutions apply to individual-level data too, yes? though I am comfortable with removing the first edit re "whether avialbale as individual or aggregate,..." - since can put that line/consideration into the discussion section?
distributional, sampling, censoring, and measurement biases.
\par
Our aim is therefore to motivate and develop bias adjuments for estimating: %SM: why is "to motivate" an aim? not sure I followed that part?
\begin{enumerate}
  \item duration in a risk group
  \item rate of partnership change
\end{enumerate}
from aggregate cross-sectional survey data, considering these factors.  %SM: in the discussion, state that can/should use these approaches when using individual-level data too :) 
We explore these topics using aggregate estimates from
a 2011 FSW survey in Eswatini~\cite{Baral2014}.

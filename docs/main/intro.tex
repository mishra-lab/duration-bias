\section{Introduction}
Mathematical models of sexually transmitted infections require
estimates of sexual behaviour for model inputs (parameters) \cite{Garnett2002}.
In risk-stratified models, two important parameters are:
the duration of time within epidemiological risk states/groups, and
rates of sexual partnership change for each group
\cite{Garnett2002,Watts2010,Boily2015,Knight2020}.
For example, the mean duration of time selling sex can be used to define
the modelled rate of ``turnover'' among sex workers \cite{Knight2020},
while, numbers of main, casual, and/or paying sexual partners per year
can be used to define the modelled ``force of infection'' \cite{Boily2015}.
\par
Ideally, these parameters could be informed by
individual-level data from cohort studies.
However, in many cases, only
aggregate estimates from published cross-sectional studies are available.
% LW: i think most the 'bias' mentioned here was less so due to aggregate
%     vs individual-level data, but rather the cross sectional nature of the studies.
%     E.g., ideally case -  if we can have a cohort study and follow everyone till they end SW
%     or till death - we could have a better estimates.
%     however- a cohort study with long term follow-up are less feasible.
%     Cross-sectional studies were lower cost - and more commonly adopted.
%     yet- subject to xyz limitations as u list here.
%     and set up your objective as given commonly collected cross-sectional data.
%     how we can improve estimates of interest.
%     Then u can add that the proposed approach can be applied in cases
%     when only aggregate estimates are available.
While much attention has been paid to
sampling and reporting biases affecting sexual behaviour data \cite{Fenton2001,Langhaug2010},
relatively less work has examined measurement error
in quantifying dynamic sexual behaviour from cross-sectional data \cite{Burington2010,Fazito2012}.
Our aim is therefore to described and address
several types of duration-related biases when estimating:
(1) mean duration selling sex, and (2) mean rates of partnership change,
from aggregate cross-sectional estimates.
We develop Bayesian hierarchical models to
integrate multiple potential mechanisms of bias, and
support inference of the unbiased parameters of interest.
We use data from a 2011 female sex worker survey in Eswatini \cite{Yam2013}
to support parameterization of an HIV transmission model.

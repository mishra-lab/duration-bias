\section{Discussion}
We sought to develop bias adjustments for estimating
mean duration in epidemiological risk states (periods of risk) and
mean rates of sexual partnership change
from aggregate cross-sectional data.
We developed these adjustments using Bayesian hierarchical models to incorporate
uncertainty in the available data and
mechanistic assumptions about several bias-generating processes.
We showed that these adjustments can influence
estimated parameter means by factors approaching 2,
suggesting that unadjusted estimates of these parameters
% LW: would prefer this term than 'biased estimates'. as i noted in my comments earlier
% JK: done -- edit: actually I prefer unadjusted (reverted)
should be interpreted with care.
\par
We grounded our study in the analysis of aggregate sex work data
to parameterize a mathematical model of HIV transmission.
However, our approach should be broadly applicable to
analysis of other intermittent risk exposures and event rates,
including analysis of individual-level data for conventional statistical models.
For example, periods of hazardous conditions may need to be quantified
in an empiric study of workplace injury risk.
Additionally, estimates of population-attributable fractions
may be improved through our observation that: in many cross-sectional studies,
reported exposure duration reflects only half of the total expected exposure duration.
\par
% LW: i think in discussion u can also talk about the gap in data collection.
%     importance of incorporte survey questions on duration of partnership etc. and on gaps
%     then mention even when those questions are asked - subject to censoring -
%     hence ur method can be applied there too
% JK: Ah, great point! I've added below
Our work highlights key variables to collect in sexual behaviour surveys, such as
numbers of sexual partners per recall period,
sexual partnership durations, and
number/durations of gaps within sexual partnerships and/or period of risk.
Such survey questions could even be designed to support
a pre-specified hierarchical model,
and use built-in redundancy to validate model assumptions,
such as multiple recall periods.
\par
Our work can be built upon by considering
further potential sources of bias and/or uncertainty.
For example, we assumed a fixed duration for each sexual partnership type,
but this duration could be modelled as another random variable
whose distribution could also be inferred from additional data.
Future work could also consider
rounding error~\cite{Mills2014},
recall bias~\cite{Ramjee1999},
reporting bias~\cite{Lowndes2012},
and the like \cite{Fenton2001}.
Finally, our approach could be extended to support
triangulation from multiple data sources \cite{Lawlor2016},
with potentially different mechanisms of bias for each source explicitly modelled.

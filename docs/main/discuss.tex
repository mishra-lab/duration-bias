\section{Discussion}
We sought to develop bias adjustments for estimating
the mean duration in a risk group (or "seasons/periods of risk") and rates of sexual partnership change
from aggregate cross-sectional data.
We developed these adjustments using Bayesian hierarchical models in order to incorporate
uncertainty in the available data and mechanistic assumptions about the bias-generating processes.
We showed that these adjustments can influence estimated variable means by factors approaching 2,
suggesting that unadjusted estimates of these variables should be interpreted carefully.
\par
We grounded our study in the analysis of
aggregate sex work data for parameterization of transmission models of HIV and other sexually transmitted infections. %SM: JK - I think we do not use STI acronym more than 3 times in the paper, so lets not use acronym? 
However, our approach should be broadly applicable to
analysis of other data sources and for other purposes, %SM: i think be a bit more specific - what other purposes? 
including analysis of individual-level data for conventional statistical models. %SM: like what? empricial studies of...? estimates of duration, given duration has been noted as a predictor of HIV/STI prevalence, incidence, etc.? I think a few concrete examples might be helpful to strengthen the take-away from this amazing work! :) 
Our work can also be built upon by considering
further potential sources of bias and/or uncertainty.
For example, we assumed a fixed duration for each sexual partnership type,
but this duration could be modelled as another random variable
whose distribution could also be inferred.
Moreover, future work could consider
rounding error~\cite{Mills2014},
recall bias~\cite{Ramjee1999},
reporting bias~\cite{Lowndes2012},
and the like \cite{Fenton2001}.

%SM: I love this section above re: future work. Do most of these papers end with that, or do we want to end with the "broadly applicable" part? 
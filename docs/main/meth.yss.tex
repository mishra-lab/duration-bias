%---------------------------------------------------------------------------------------------------
\paragraph{Crude Estimates}
The survey \cite{Baral2014} included questions about
the current respondent's age and the age of first selling sex.
The difference between these ages could be used to define a crude ``duration selling sex''.
Using this approach, the crude median duration was $\tilde{d} = 4$ years.
However, if durations are assumed to be exponentially distributed
--- a implicit assumption in compartmental models \cite{Anderson1991} ---
then the crude mean could be estimated from the crude median as
$\bar{d} = \tilde{d}/\distr{log}{2}$ due to skewness.
To move beyond crude estimates, next we develop the hierarchical model,
considering the following potential biases.
%---------------------------------------------------------------------------------------------------
\paragraph{Sampling}
Sampling bias was considered via RDS-adjustment in \cite{Baral2014},
yielding mean and \ci estimates of the proportions of respondents $p_z$
who had sold sex starting $\mathbb{d}_z \in \{0{-}2, 3{-}5, 6{-}10, 11+\}$ years ago
(Table~\ref{tab:data}, ``$z$'' enumerates strata).
We start by defining a model to identify distributions of reported durations $d_i$
which are consistent with these data.
We model each proportion $p_z$ as a random variable with
a beta approximation of binomial (BAB) distribution (see Appendix~\ref{app.bab})
with parameters $N_z$ and $\rho_z$.
We model each $N_z$ as a fixed value,
which we fit to the \ci of $p_z$ as described in \sref{app.bab}.
We then model each $\rho_z$ as
the proportion of reported durations $d_i$ within the interval $\mathbb{d}_z$.
Since these proportions are difficult to define analytically,
we estimate $\hat{\rho}_z = \distr{mean}{d_i \in \mathbb{d}_z}$ from $N = 100$ samples.
%---------------------------------------------------------------------------------------------------
\paragraph{Censoring}
These reported durations $d_i$ are effectively right censored
because they only capture engagement in sex work up until the survey, and
and not additional sex work after the survey
(Figure~\ref{fig:diag.yss.censor}) \cite{Fazito2012}.
If we assume that the survey reaches respondents at a random time point
during their total (eventual) duration selling sex $D_i$, we can model this censoring via
a random fraction $f_i \sim \distr{Unif}{0,1}$, such that $d_i = f_i D_i$;
the expected means are then related by $\bar{d} / \bar{D} = \bar{f} = \frac12$ \cite{Tufto2017}.
%---------------------------------------------------------------------------------------------------
\paragraph{Measurement}
Finally, respondents may not sell sex continuously.
Reported durations $d_i$ may therefore include
multiple periods of selling sex with gaps in between,
whereas we aim to model $D_i$ as the durations of individual periods selling sex.
Respondents in \cite{Baral2014} were not asked whether they ever temporarily stopped selling sex,
but a later survey \cite{EswKP2014} indicated that $\phi = 45\%$ had stopped at least once.
We model the number of times a respondent may temporarily stop selling sex as
a Poisson-distributed random variable $s_i$ with mean~$\bar{s}$.
The expected value of $\phi$ given $\bar{s}$ is then $\distr{P}{s > 0} = 1 - e^{-\bar{s}}$.
Since $\phi = 45\%$ is an imperfect observation,
we model $\phi$ as a random variable with a BAB distribution
having parameters $N = 328$ and $\rho = 1 - e^{-\bar{s}}$,
which allows inference on $\bar{s}$ given $\phi$.
\par
Next, we update the model for reported durations as $d_i = D_i\,(f_i + s_i\,(1 + g_i))$,
where $g_i$ is the relative duration of gaps between selling sex,
with the following rationale.
If $s_i = 0$, then $d_i = f_i D_i$ as before, reflecting the censored current period only.
If $s_i > 0$, then $d_i$ also includes $s_i$ prior periods selling sex and the gaps between them
(Figure~\ref{fig:diag.yss.gaps}) --- \ie $s_i\,(D_i + g_i D_i) = D_i\,s_i\,(1 + g_i)$.
The major assumption we make here is that
all successive periods are of equal length, and likewise for gaps between them.
We must also assume a distribution for $g_i$, for which we choose
$g_i \sim \distr{Exp}{1/\bar{g}}$, arbitrarily.
%---------------------------------------------------------------------------------------------------
\paragraph{Summary}
Figure~\ref{fig:model.yss} summarizes the proposed model graphically.
The primary parameter of interest is
the mean duration selling sex (for a given period) $\bar{D}$,
but we must also infer
the mean number of times respondents stop selling sex $\bar{s}$, and
the mean relative duration of gaps $\bar{g}$.
We assume uninformative priors for these 3 parameters.
